\pretextualchapter{Resumo}

\fazreferencia

% Indicativo do trabalho:
% Entendimento do trabalho sem precisar ler o trabalho todo
% somente um parágrafo de 150 a 500 palavras segundo abnt,
% frases afirmativas curtas em voz ativa, 3 pessoa singular

% O que é (contexto)
Este trabalho tem como objetivo desenvolver um banco de dados que atenda ao acervo do Laboratório de Fotogrametria e Sensoriamento Remoto (LFSR) da Universidade do Estado do Rio de Janeiro (UERJ).
%relevância
Sua justificativa se dá pela carência na organização, armazenamento persistente em meio digital e acessibilidade aos dados brutos do acervo e também aos resultantes de projetos de mapeamento fotogramétrico realizados no LFSR da UERJ.
% Como? (Método utilizado)
A metodologia utilizada é composta por um levantamento de requisitos; uma modelagem UML segundo o paradigma de Orientação a Objetos (OO); a implementação de código-fonte; a realização da carga e testes de usabilidade. No caso particular de carga e visualização das imagens, foram utilizados scripts em python e o software QGIS.
% Resultados:
Os resultados incluem uma instância de banco de dados objeto-relacional implementada num servidor postgresql, com sua extensão espacial (postgis), para quatro conjuntos de projetos fotogramétricos do LFSR. Os testes realizados incluíram a elaboração de buscas para verificar a implementação e a validação quanto ao atendimento dos requisitos. Tais testes mostraram que o modelo pode ser adotado para diversos projetos simultâneos, permitindo sua comparação e o reaproveitamento de informações.
% Por quê? (Descreva o objetivo do trabalho):
Uma das conclusões do trabalho foi a de que o banco de dados proposto servirá para armazenar, em meio digital, os dados disponíveis que, atualmente, encontram-se em estado físico ou `\textit{hardcopy}', bem como para organizar dados digitais de projetos pré-existentes LFSR. Espera-se que este banco também possa vir a ser usado como ferramenta de aprendizagem para os alunos do curso de Engenharia Cartográfica, tanto nas matérias de Fotogrametria e Sensoriamento Remoto, quanto nas disciplinas relacionadas à Computação Aplicada à Cartografia.
% Conclui-se que:
Espera-se também que o modelo possa servir de base para novos projetos e, quando ampliado, para contemplar áreas não desenvolvidas até o momento por este laboratório, tal como uso de imagens orbitais ou a sua integração com a Infraestrutura Nacional de Dados Espaciais (INDE) ou com a restituição estereofotogramétrica realizada diretamente no padrão da Estrutura de Dados Geoespaciais Vetoriais (EDGV).
% Onde os resultados foram disponibilizados:
Os resultados deste trabalho estão disponíveis em \url{https://github.com/T4tyana/TCC}.
 
\imprimirchaves