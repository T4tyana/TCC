\pretextualchapter{Abstract}

\doreference

%%%
This work aims to the development of a database for organizing the airborne imagery collection of the Photogrammetry and Remote Sensing Laboratory (LFSR) of the Rio de Janeiro State University (UERJ).
%%%
The proposed database will be used to the permanent storing in digital media of photogrammetric data, currently available only in physical or `hardcopy ' format. 
%%%
The modeling was developed using the UML, following the OO modeling paradigm; the implementation of the source code; the database uploading, and the usability tests. Python scripts and QGIS software were used for the visualization of in the particular case of photogrammetric imagery.
%%%
As a result of this work a relational-object database was generated and implemented with a postgresql/postgis software tool. Its initial loading was composed by four basic projects obtained through the LFSR. The tests for checking and validation of this database have shown that the model is effective and efficient. The tests also showed that the developed model could be used for many projects simultaneously, thus permitting comparisons between data sets, as well as reusing of stored information. 
%%%
One of the conclusions is that the proposed database will serve for its intended goals.
%%%
One expects that this database could be used as a learning tool for the subjects of photogrammetry and remote sensing, as well as for the computer aided cartography.
One expects that this model will serve as a base for new projects and when extended, for encompassing of new photogrammetric data, such as spaceborne imagery, and also for the integration with both the National Spatial Data Infrastructure (INDE) and the Vector Geospatial Data Structure (EDGV) brazilian data structures.
%%%
The results of this work are available at \url{https://github.com/T4tyana/TCC}.












\printkeys