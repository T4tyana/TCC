%---------------------------------------------------------------------
% Imagens pretextuais (precisam estar no mesmo diretório deste arquivo .tex)
%---------------------------------------------------------------------
\logo{formato/logo_uerj_cinza.png}
\marcadagua{formato/marcadagua_uerj_cinza.png}{1}{160}{255}

%---------------------------------------------------------------------
% Informações da instituição
%---------------------------------------------------------------------
\instituicao{Universidade do Estado do Rio de Janeiro}
            {Centro de Tecnologia e Ciências} 
            {Faculdade de Engenharia} 
            {Departamento de Engenharia Cartográfica} 
             

%---------------------------------------------------------------------
% Informações da autoria do documento
%---------------------------------------------------------------------
\autor{Tatyana Vargas Queiroz} % Nome e sobrenomes do meio 
      {Vieira} % Último nome
      {T. V. Q.}      % Abreviação dos nomes iniciais

\titulo{Desenvolvimento de um Banco de Dados para o Laboratório de Fotogrametria e Sensoriamento Remoto}
\palavraschaves{Cartografia}
               {Banco de Dados Geoespaciais}
               {Fotogrametria}
               {Sensoriamento Remoto}
               {} % se não for usar a quarta palavra chave, deixar o campo vazio: {}

\title{Development of a Database for the Photogrammetry and Remote Sensing Laboratory}
\keywords{Cartography}
         {Geospatial Database}
         {Photogrametry}
         {Remote Sensing}

\orientador{Prof. Ph.D.} 
           {Jorge Luís}{Nunes e Silva Brito} 
           {Faculdade de Engenharia -- UERJ} 

%coorientador é opcional
\coorientador{Prof. Me.} 
             {Irving}{da Silva Badolato}
             {Faculdade de Engenharia -- UERJ} 

%---------------------------------------------------------------------
% Grau pretendido (Doutor, Mestre, Graduado, Bacharel, Licenciado, Engenheiro), % Curso (Engenharia Cartográfica, Engenharia Ambiental, Engenharia Elétrica) e 
% Gênero (Masculino, Feminino)
%---------------------------------------------------------------------
\grau{Engenheiro}  
\curso{Engenharia Cartográfica}
\genero{Feminino}

%---------------------------------------------------------------------
% Informações adicionais (local, data e paginas)
%---------------------------------------------------------------------
\local{Rio de Janeiro} 
\data{10}{12}{2019} % <= Atualize com a data da defesa
