% Pacotes fundamentais 
\usepackage[utf8]{inputenc} % Determina a codificação utilizada
                            % (conversão automática dos acentos)
\usepackage[brazil]{babel}  % adequação para o português Brasil
\usepackage{makeidx}        % Cria o índice
\usepackage{hyperref}       % Controla a formação do índice
\usepackage{indentfirst}    % Endenta o primeiro paragrafo de
                            % cada seção.
\usepackage{graphicx}       % Inclusão de gráficos
\usepackage{subfig}
\usepackage{amsmath}        % pacote matemático
\usepackage{listings}       % Include the listings-package
\usepackage{microtype}
\usepackage{gensymb}
\usepackage{xcite}
\usepackage{float}
\usepackage{times}
\usepackage{tikz}
\usepackage{pdflscape}      % Recurso para páginas em paisagem. Útil nos anexos de grandes imagens.
\usetikzlibrary{fadings}

% Pacote auxiliar para as normas da UERJ
\usepackage[frame=no,algline=yes,font=default]{formato/repUERJformat}
\usepackage{formato/repUERJpseudocode}

% Pacotes de citações
\usepackage[alf]{abntex2cite}
