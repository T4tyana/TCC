\chapter*{Conclusão}
%===================================================================== 
A motivação para a realização deste projeto foi originada a partir da observação das dificuldades de acesso aos dados presentes no acervo do Laboratório de Fotogrametria e Sensoriamento Remoto (LFsR) do Departamento de Engenharia Cartográfica da UERJ, bem como a possibilidade de perda de parte desses dados, devido à sua forma de armazenamento em meio analógico. Assim sendo, foi buscada uma nova forma de armazenar tais dados em caráter permanente provendo uma melhor organização e acesso aos mesmos. O projeto contido neste trabalho apresentou a solução idealizada para o problema exposto. 

Este projeto teve como objetivo a criação de um banco de dados para atender às necessidades do LFSR. A ideia era de que a documentação gerada pudesse replicar o processo de criação desse banco de dados em um servidor de escolha do LFSR, possibilitando o armazenamento do acervo de dados do laboratório. Prevendo o potencial do banco de dados resultante se tornar obsoleto, o mesmo foi modelado para que englobasse também as fases que compunham a etapa de inicialização de um projeto fotogramétrico.

%Como exposto ao longo dos capítulos anteriores, decidiu-se focar no processo fotogramétrico digital que trabalhasse com dados originados da aerofotogrametria tradicional (imagens analógicas digitalizadas). Através da carga de um banco de dados piloto, foi visto que este modelo atende tanto aos dados de origem analógica quanto aos dados digital.

De acordo com os questionamentos e resultados apresentados no capítulo \ref{results} pode-se concluir que o objetivo estabelecido foi alcançado. Este fato pode ser comprovado pela verificação e validação do projeto piloto implementado. Por fim, o item \ref{testefinal} informa a réplica bem sucedida do processo estabelecido neste trabalho por intermédio da documentação gerada pelo mesmo. Logo, conclui-se que o banco, fruto deste processo, é passível de ser implementado em outras máquinas.  

Como sugestões para trabalhos futuros apresentam-se as seguintes:

\begin{itemize}
    \item Melhorias que englobem não somente um processo fotogramétrico no nível aéreo, mas também no nível orbital são esperada;
    \item A criação de uma interface para o banco de dados, que seja mais amigável ao usuário e;
    \item Ampliação do modelo para a integração com a INDE e para possibilitar a execução da restituição estereofotogramétrica diretamente no padrão EDGV.
\end{itemize}
