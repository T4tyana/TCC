\chapter*{introdução}

%NOTA: O texto que segue, com comentários (em latex) é importante exemplo baseado em trabalho de aluno anterior do Carto. Observe o modelo de texto em ch0.tex e edite-o para seu projeto final de curso adicionando sempre que possível perguntas relacionadas ao seu tema, como comentários, para orientar sua escrita e documentar o processo de elaboração de cada parágrafo. Tais comentários ajudam o aluno, a escrever, e podem ser complementadas no futuro com demais questões impostas pelo orientador ou pela banca. 

%O que são dados?
Todo trabalho, estudo ou desenvolvimento tecnológico possui objetivos, resultados e utiliza um método. Uma das partes mais importantes de um esforço de desenvolvimento científico-tecnológico são os dados. Na realidade os dados estão presentes: no início, sobre os quais a hipótese científica será testada; no momento em que esses dados são analisados e trabalhados; e no fim, onde os dados finais obtidos são chamados de resultados. Assim, pode-se dizer que `dado' é todo e qualquer insumo utilizado na realização de um trabalho. 
%Qual a importância da organização de dados?
Por outro lado, dado por si só é um fato isolado que pode ou não ser relevante para um projeto. Sua organização permite que o pesquisador entenda qual é sua viabilidade e a forma que melhor se adéqua ao desenvolvimento do trabalho. 

A sociedade atual vive o que se chama de `Era Digital', onde programas de computador são capazes de processar inimagináveis volumes de dados, de forma rápida e eficiente e, preservando esses dados em caráter permanente. O meio digital pode ser reconhecido como uma excelente forma de armazenamento de dados.

% Qual é o objeto de estudo deste trabalho?
Este trabalho tem por objetivo geral o desenvolvimento de um banco de dados para a organização do acervo técnico do Laboratório de Fotogrametria e Sensoriamento Remoto (LFSR) do Departamento de Engenharia Cartográfica da Universidade Estadual do Rio de Janeiro - UERJ. 

Como objetivos específicos, listam-se os seguintes:
\begin{itemize}
\item Revisão da literatura técnica sobre bancos de dados, com ênfase em bancos de dados geoespaciais;
\item Seleção do modelo conceitual a ser utilizado;
\item Levantamento dos requisitos para o banco a ser desenvolvido;
\item Modelagem do banco de dados, segundo a abordagem orientada a objetos;
\item Implementação do banco segundo a metodologia ágil;
\item Armazenamento do banco de dados;
\item Testes do banco de dados, por intermédio do desenvolvimento de consultas típicas, baseado no levantamento de requisitos.
\end{itemize}

% Qual a importância do laboratório no departamento de cartografia?
O LFSR é essencial para o processo ensino-aprendizagem dos alunos do curso de graduação de Engenharia Cartográfica. Além disso é importante para a disseminação do conhecimento relevante à disciplina de Fotogrametria Digital, para alunos externos e pesquisadores da área. O acervo do LFSR possui inúmeros dados, como fotogramas que se apresentam em formato analógico (em papel fotográfico), o que vem dificultando a organização e utilização dos mesmos. Grande parte do acervo do laboratório pode ser aproveitado para futuros projetos dos alunos do curso, porém o acervo fotográfico do Laboratório, em formato de papel, tem implicado na perda desses dados por efeito do tempo e das condições de armazenamento do material.

% Como isso se encaixa no curso de Cartografia?
A Engenharia Cartográfica, por sua vez, tem como objetivo geral o mapeamento de uma área geográfica de interesse. Uma das formas de realização dessa tarefa é através da utilização de técnicas de mapeamento Fotogramétrico. Com o atual estágio do desenvolvimento científico-tecnológico, no contexto da Engenharia Cartográfica, os produtos do mapeamento, de uma maneira geral, são realizados por intermédio de computadores digitais, dando origem ao conceito de dados e informações geoespaciais. Essas informações necessitam ser organizadas e armazenadas, em estruturas de dados vetoriais e matriciais, requerendo, na maior parte das vezes, a análise, o projeto e a implementação de um banco de dados geoespaciais. Este fato demonstra, com clareza, a importância do conceito e a aplicabilidade de um banco de dados na área de Engenharia Cartográfica.

%Organização do dados - qual é a necessidade de um banco?
A solução idealizada neste trabalho, para organização dos dados em meio digital, requer a análise de requisitos, projeto e implementação de um banco de dados modelado a partir das necessidades do laboratório. Além disso fez-se necessário entender o potencial do uso deste banco como uma nova ferramenta, não só de armazenamento, mas também de aprendizado para as disciplinas que fazem uso desse laboratório. 
O projeto piloto do banco engloba grande parte do processo aerofotogramétrico, e deixa, em aberto, possibilidades para futuras expansões que venham a incluir processos de dados de sensoriamento remoto além dos obtidos pela fotogrametria aérea tradicional.

%Dados fotogramétricos:
O banco de dados proposto, como dito previamente, foca na fotogrametria, cujo objetivo principal corresponde a reconstrução de um espaço 3D, a partir de imagens 2D, de forma a representar, georreferenciar e medir o objeto imageado da forma mais fiel possível. Um banco de dados fotogramétrico, por sua vez, armazena todas as informações pertinentes à aquisição das imagens e seu subsequente processamento, de forma que o projeto fotogramétrico seja preservado e reutilizado. Essa linha de pensamento será desenvolvida ao longo do trabalho. 

% Qual é a organização deste trabalho? 
Este trabalho é estruturado em três capítulos, além da Introdução, Conclusão, Referências, Apêndices e Anexos. O primeiro consiste na fundamentação teórica que dê ao leitor o embasamento necessário para que o mesmo seja capaz de entender como este trabalho foi realizado. O segundo capítulo trata da metodologia, e apresenta como o modelo proposto foi desenvolvido, desde um primeiro momento, onde ocorre o levantamento de requisitos, até a geração do código base, usado para a implementação do banco de dados piloto. O terceiro capítulo, por sua vez, apresenta os resultados gerados e testes do projeto piloto, bem como a discussão proveniente dos mesmos. Por fim, uma breve conclusão e referências bibliográficas são apresentadas.